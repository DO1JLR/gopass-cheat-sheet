\PassOptionsToPackage{unicode=true}{hyperref} % options for packages loaded elsewhere
\PassOptionsToPackage{hyphens}{url}
\PassOptionsToPackage{dvipsnames,svgnames*,x11names*}{xcolor}
%
\documentclass[9pt,english,a4paper,]{scrartcl}
\usepackage{lmodern}
\usepackage{amssymb,amsmath}
\usepackage{ifxetex,ifluatex}
\usepackage{fixltx2e} % provides \textsubscript
\ifnum 0\ifxetex 1\fi\ifluatex 1\fi=0 % if pdftex
  \usepackage[T1]{fontenc}
  \usepackage[utf8]{inputenc}
  \usepackage{textcomp} % provides euro and other symbols
\else % if luatex or xelatex
  \usepackage{unicode-math}
  \defaultfontfeatures{Ligatures=TeX,Scale=MatchLowercase}
\fi
% use upquote if available, for straight quotes in verbatim environments
\IfFileExists{upquote.sty}{\usepackage{upquote}}{}
% use microtype if available
\IfFileExists{microtype.sty}{%
\usepackage[]{microtype}
\UseMicrotypeSet[protrusion]{basicmath} % disable protrusion for tt fonts
}{}
\IfFileExists{parskip.sty}{%
\usepackage{parskip}
}{% else
\setlength{\parindent}{0pt}
\setlength{\parskip}{6pt plus 2pt minus 1pt}
}
\usepackage{xcolor}
\usepackage{hyperref}
\hypersetup{
            pdftitle={Gopass Cheat Sheet},
            pdfauthor={Santiago Fraire Willemoes},
            pdfkeywords={gpg, crypto, password, secrets},
            colorlinks=true,
            linkcolor=blue,
            citecolor=Blue,
            urlcolor=Blue,
            breaklinks=true}
\urlstyle{same}  % don't use monospace font for urls
\usepackage[a4paper,left=1cm,right=1cm,top=1cm,bottom=1cm]{geometry}
\usepackage{color}
\usepackage{fancyvrb}
\newcommand{\VerbBar}{|}
\newcommand{\VERB}{\Verb[commandchars=\\\{\}]}
\DefineVerbatimEnvironment{Highlighting}{Verbatim}{commandchars=\\\{\}}
% Add ',fontsize=\small' for more characters per line
\newenvironment{Shaded}{}{}
\newcommand{\AlertTok}[1]{\textcolor[rgb]{1.00,0.00,0.00}{\textbf{#1}}}
\newcommand{\AnnotationTok}[1]{\textcolor[rgb]{0.38,0.63,0.69}{\textbf{\textit{#1}}}}
\newcommand{\AttributeTok}[1]{\textcolor[rgb]{0.49,0.56,0.16}{#1}}
\newcommand{\BaseNTok}[1]{\textcolor[rgb]{0.25,0.63,0.44}{#1}}
\newcommand{\BuiltInTok}[1]{#1}
\newcommand{\CharTok}[1]{\textcolor[rgb]{0.25,0.44,0.63}{#1}}
\newcommand{\CommentTok}[1]{\textcolor[rgb]{0.38,0.63,0.69}{\textit{#1}}}
\newcommand{\CommentVarTok}[1]{\textcolor[rgb]{0.38,0.63,0.69}{\textbf{\textit{#1}}}}
\newcommand{\ConstantTok}[1]{\textcolor[rgb]{0.53,0.00,0.00}{#1}}
\newcommand{\ControlFlowTok}[1]{\textcolor[rgb]{0.00,0.44,0.13}{\textbf{#1}}}
\newcommand{\DataTypeTok}[1]{\textcolor[rgb]{0.56,0.13,0.00}{#1}}
\newcommand{\DecValTok}[1]{\textcolor[rgb]{0.25,0.63,0.44}{#1}}
\newcommand{\DocumentationTok}[1]{\textcolor[rgb]{0.73,0.13,0.13}{\textit{#1}}}
\newcommand{\ErrorTok}[1]{\textcolor[rgb]{1.00,0.00,0.00}{\textbf{#1}}}
\newcommand{\ExtensionTok}[1]{#1}
\newcommand{\FloatTok}[1]{\textcolor[rgb]{0.25,0.63,0.44}{#1}}
\newcommand{\FunctionTok}[1]{\textcolor[rgb]{0.02,0.16,0.49}{#1}}
\newcommand{\ImportTok}[1]{#1}
\newcommand{\InformationTok}[1]{\textcolor[rgb]{0.38,0.63,0.69}{\textbf{\textit{#1}}}}
\newcommand{\KeywordTok}[1]{\textcolor[rgb]{0.00,0.44,0.13}{\textbf{#1}}}
\newcommand{\NormalTok}[1]{#1}
\newcommand{\OperatorTok}[1]{\textcolor[rgb]{0.40,0.40,0.40}{#1}}
\newcommand{\OtherTok}[1]{\textcolor[rgb]{0.00,0.44,0.13}{#1}}
\newcommand{\PreprocessorTok}[1]{\textcolor[rgb]{0.74,0.48,0.00}{#1}}
\newcommand{\RegionMarkerTok}[1]{#1}
\newcommand{\SpecialCharTok}[1]{\textcolor[rgb]{0.25,0.44,0.63}{#1}}
\newcommand{\SpecialStringTok}[1]{\textcolor[rgb]{0.73,0.40,0.53}{#1}}
\newcommand{\StringTok}[1]{\textcolor[rgb]{0.25,0.44,0.63}{#1}}
\newcommand{\VariableTok}[1]{\textcolor[rgb]{0.10,0.09,0.49}{#1}}
\newcommand{\VerbatimStringTok}[1]{\textcolor[rgb]{0.25,0.44,0.63}{#1}}
\newcommand{\WarningTok}[1]{\textcolor[rgb]{0.38,0.63,0.69}{\textbf{\textit{#1}}}}
\setlength{\emergencystretch}{3em}  % prevent overfull lines
\providecommand{\tightlist}{%
  \setlength{\itemsep}{0pt}\setlength{\parskip}{0pt}}
\setcounter{secnumdepth}{0}
% Redefines (sub)paragraphs to behave more like sections
\ifx\paragraph\undefined\else
\let\oldparagraph\paragraph
\renewcommand{\paragraph}[1]{\oldparagraph{#1}\mbox{}}
\fi
\ifx\subparagraph\undefined\else
\let\oldsubparagraph\subparagraph
\renewcommand{\subparagraph}[1]{\oldsubparagraph{#1}\mbox{}}
\fi
\pagestyle{empty}

% set default figure placement to htbp
\makeatletter
\def\fps@figure{htbp}
\makeatother

\ifnum 0\ifxetex 1\fi\ifluatex 1\fi=0 % if pdftex
  \usepackage[shorthands=off,main=english]{babel}
\else
  % load polyglossia as late as possible as it *could* call bidi if RTL lang (e.g. Hebrew or Arabic)
  \usepackage{polyglossia}
  \setmainlanguage[]{english}
\fi

\title{Gopass Cheat Sheet}
\author{Santiago Fraire Willemoes}
\date{}

\usepackage{multicol}

\let\oldtexttt\texttt
\renewcommand{\texttt}[1]{{\color{blue}{\oldtexttt{#1}}}}



\begin{document}

\raggedright

\footnotesize

\begin{multicols}{3}
\hypertarget{gopass-cheat-sheet}{%
\section{GOPASS CHEAT SHEET}\label{gopass-cheat-sheet}}

Secure passwords with \href{https://gopass.pw}{gopass}. It creates a
folder tree, where encrypted files are the leaves.

\begin{Shaded}
\begin{Highlighting}[]
\ExtensionTok{gopass}
\NormalTok{├── }\ExtensionTok{my-company}
\NormalTok{│   └── }\ExtensionTok{pepe@my-company.com}
\NormalTok{└── }\ExtensionTok{personal}
\NormalTok{    └── }\ExtensionTok{pepe@personal.com}
\end{Highlighting}
\end{Shaded}

\hypertarget{gpg-keys}{%
\subsection{GPG Keys}\label{gpg-keys}}

\hypertarget{list-secret-keys}{%
\subsubsection{List secret keys}\label{list-secret-keys}}

\begin{Shaded}
\begin{Highlighting}[]
\ExtensionTok{gpg}\NormalTok{ -K}
\end{Highlighting}
\end{Shaded}

\hypertarget{create-new-key-required}{%
\subsubsection{Create new key
(required)}\label{create-new-key-required}}

\begin{Shaded}
\begin{Highlighting}[]
\ExtensionTok{gpg}\NormalTok{ --full-generate-key}
\end{Highlighting}
\end{Shaded}

\hypertarget{initialize-gopass}{%
\subsection{Initialize gopass}\label{initialize-gopass}}

\hypertarget{autocomplete}{%
\subsubsection{Autocomplete}\label{autocomplete}}

\begin{Shaded}
\begin{Highlighting}[]
\BuiltInTok{echo} \StringTok{"source <(gopass completion bash)"} \OperatorTok{>>}\NormalTok{ ~/.bashrc}
\end{Highlighting}
\end{Shaded}

\hypertarget{initialize-new-password-store-required}{%
\subsubsection{Initialize new password store
(required)}\label{initialize-new-password-store-required}}

\begin{Shaded}
\begin{Highlighting}[]
\ExtensionTok{gopass}\NormalTok{ init}
\end{Highlighting}
\end{Shaded}

Note: backup your private key in an encrypted disk.

\hypertarget{using-gopass}{%
\subsection{Using gopass}\label{using-gopass}}

\hypertarget{list-passwords}{%
\subsubsection{List passwords}\label{list-passwords}}

\begin{Shaded}
\begin{Highlighting}[]
\ExtensionTok{gopass}\NormalTok{ ls}
\end{Highlighting}
\end{Shaded}

\hypertarget{creating-passwords}{%
\subsubsection{Creating passwords}\label{creating-passwords}}

Default store location \texttt{\textasciitilde{}/.password-store/}

\begin{Shaded}
\begin{Highlighting}[]
\ExtensionTok{gopass}\NormalTok{ insert my-company/willy@email.com}
\end{Highlighting}
\end{Shaded}

\hypertarget{generate-random-pass}{%
\subsubsection{Generate random pass}\label{generate-random-pass}}

\begin{Shaded}
\begin{Highlighting}[]
\ExtensionTok{gopass}\NormalTok{ generate my-company/anothername@rmail.com}
\end{Highlighting}
\end{Shaded}

\hypertarget{search-secrets}{%
\subsubsection{Search secrets}\label{search-secrets}}

\begin{Shaded}
\begin{Highlighting}[]
\ExtensionTok{gopass}\NormalTok{ search @email.com}
\end{Highlighting}
\end{Shaded}

\hypertarget{show-password-in-console}{%
\subsubsection{Show password in
console}\label{show-password-in-console}}

\begin{Shaded}
\begin{Highlighting}[]
\ExtensionTok{gopass}\NormalTok{ my-company/willy@email.com}
\end{Highlighting}
\end{Shaded}

\hypertarget{copy-password-to-clipboard}{%
\subsubsection{Copy password to
clipboard}\label{copy-password-to-clipboard}}

\begin{Shaded}
\begin{Highlighting}[]
\ExtensionTok{gopass}\NormalTok{ -c my-company/willy@email.com}
\end{Highlighting}
\end{Shaded}

\hypertarget{using-stores}{%
\subsection{Using stores}\label{using-stores}}

Stores (AKA mounts) let you group your passwords. Example:
\texttt{personal}, \texttt{company}. Each one can live in a different
repository, and you could share \texttt{company} with your peers.

\hypertarget{initialize-new-store}{%
\subsubsection{Initialize new store}\label{initialize-new-store}}

Creates a new store located at
\texttt{\textasciitilde{}/.password-store-my-company}.

\begin{Shaded}
\begin{Highlighting}[]
\ExtensionTok{gopass}\NormalTok{ init --store my-company}
\end{Highlighting}
\end{Shaded}

\hypertarget{add-git-remote-to-store}{%
\subsubsection{Add git remote to store}\label{add-git-remote-to-store}}

\begin{Shaded}
\begin{Highlighting}[]
\ExtensionTok{gopass}\NormalTok{ git remote add --store my-company origin git@gh.com/Woile/keys.git}
\end{Highlighting}
\end{Shaded}

\hypertarget{clone-existing-store}{%
\subsubsection{Clone existing store}\label{clone-existing-store}}

\begin{Shaded}
\begin{Highlighting}[]
\ExtensionTok{gopass}\NormalTok{ clone git@gh.com/Woile/keys.git my-company --sync gitcli}
\end{Highlighting}
\end{Shaded}

\hypertarget{synchronization}{%
\subsection{Synchronization}\label{synchronization}}

\hypertarget{synchronize-with-remotes}{%
\subsubsection{Synchronize with
remotes}\label{synchronize-with-remotes}}

\begin{Shaded}
\begin{Highlighting}[]
\ExtensionTok{gopass}\NormalTok{ sync}
\end{Highlighting}
\end{Shaded}

\hypertarget{synchronzing-a-single-store}{%
\subsubsection{Synchronzing a single
store}\label{synchronzing-a-single-store}}

\begin{Shaded}
\begin{Highlighting}[]
\ExtensionTok{gopass}\NormalTok{ sync --store my-company}
\end{Highlighting}
\end{Shaded}

\hypertarget{team-sharing}{%
\subsection{Team sharing}\label{team-sharing}}

\hypertarget{export-public-key}{%
\subsubsection{Export public key}\label{export-public-key}}

\begin{Shaded}
\begin{Highlighting}[]
\ExtensionTok{gpg}\NormalTok{ -a --export willy@email.com }\OperatorTok{>}\NormalTok{ willy.pub.asc}
\end{Highlighting}
\end{Shaded}

\hypertarget{check-current-recipients}{%
\subsubsection{Check current
recipients}\label{check-current-recipients}}

\begin{Shaded}
\begin{Highlighting}[]
\ExtensionTok{gopass}\NormalTok{ recipients}
\end{Highlighting}
\end{Shaded}

\hypertarget{add-public-key-into-gopass}{%
\subsubsection{Add public key into
gopass}\label{add-public-key-into-gopass}}

\begin{Shaded}
\begin{Highlighting}[]
\ExtensionTok{gpg}\NormalTok{ --import }\OperatorTok{<}\NormalTok{ willy.pub.asc}
\end{Highlighting}
\end{Shaded}

\begin{Shaded}
\begin{Highlighting}[]
\ExtensionTok{gpg}\NormalTok{ --list-keys}
\end{Highlighting}
\end{Shaded}

\begin{Shaded}
\begin{Highlighting}[]
\ExtensionTok{gopass}\NormalTok{ recipients add willy@email.com}
\end{Highlighting}
\end{Shaded}

\href{https://github.com/Woile/gopass-cheat-sheet}{source} \textbar{}
\href{https://woile.github.io/gopass-presentation/}{presentation}
\textbar{} \href{https://twitter.com/santiwilly}{santiwilly}
\end{multicols}

\end{document}
